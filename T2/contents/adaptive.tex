\documentclass[main.tex]{subfiles}

\begin{document}

\section{Adaptive Software Development}

Adaptive software development (ASD) is an iterative, circular process of software development proposed by Jim Highsmith and Sam Bayer as a technique for building complex software and systems published in 2000.
It's focused on rapid creation and evolution of software systems and is a predecessor of modern agile development techniques.

\subsection{The Process of Adaptive Software Development}

Adaptive replaces old waterfall-like models with a repeating series of \emph{speculate}, \emph{collaborate} and \emph{learn} cycles.
It is characterized by constant change, re-evaluation, peering into an uncertain future and intense collaboration among developers, testers and customers.

\subsubsection{Speculation}

This is the setup phase. During it, developers attempt to understand the exact nature of the product and requirements of the users and customer.
Because the process is inherently circular, a previous cycle feeds this phase with new information such as bugs, user reports and new requirements found during development by the team and the customer.

Speculation comprises \emph{project initiation} and \emph{risk-driven cycle planning}.

\paragraph{Project Initiation} The project's constraints and \textit{mission statement} are set by the customer, while the team determines the project's organization, gathers more requirements, and estimates size and scope.

The \emph{mission statement} needs to be specific, yet flexible and focused. Attempting to excel in multiple dimensions usually results in a product being mediocre in all of them and excellent in none.

\paragraph{Risk-Driven Cycle Planning} The mission statement's aim is to expose the customer's desired result for the project. Most likely the customer has no knowledge of the field, so this can be done adequately in layman's terms. Using the mission statement, the team forces it's focus on a single area --- such as features, schedule, defects, or resources --- in which it must excel in the development phase.

Finally, the team fixes a set of features to implement in the next iteration and also \emph{fixes} a certain amount of time for it.

\subsubsection{Collaboration}

This is the development phase. Parallel efforts include (minor) adjustments due to changing technologies, requirements, stakeholders

\end{document}