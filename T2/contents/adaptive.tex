\documentclass[main.tex]{subfiles}

\begin{document}

\section{Adaptive Software Development}

Adaptive software development (ASD) is an iterative, circular process of software development proposed by Jim Highsmith and Sam Bayer as a technique for building complex software and systems published in 2000.
It's focused on rapid creation and evolution of software systems and is a predecessor of modern agile development techniques.

ASD derives from a conceptual perspective based on the Complex Adaptive Systems (CAS) theory and a practical perspective that supports itself on the Rapid Application Development techniques and deterministic methodologies. From CAS, it inherits its Emergence and Complexity.

The ASD cycle has six basic characteristics:
\begin{enumerate}
	\item Mission focused
	\item Feature based
	\item Iterative
	\item Time-boxed
	\item Risk driven
	\item Change tolerant
\end{enumerate}

\subsection{The Process of Adaptive Software Development}

Adaptive replaces the traditional waterfall model with a repeating series of \emph{speculate}, \emph{collaborate} and \emph{learn} cycles.
It is characterized by constant change, re-evaluation, peering into an uncertain future and intense collaboration among developers, testers and customers in the various phases.

\subsubsection[Speculation]{1. Speculation}

This is the setup phase. During it, developers attempt to understand the exact nature of the product's requirements --- what features need to be added, what problems need to be fixed, and what changes need to be made.

To facilitate this, a previous cycle feeds information to this phase, namely bugs, user reports, new requirements found during developments and finished or unfinished features.

Let's see how \emph{speculation} comprises \emph{project initiation} and \emph{risk-driven cycle planning}.

\paragraph{Project Initiation} The project's constraints and \textit{mission statement} are set by the customer. The development team determines the project's organization, gathers more requirements, and estimates size and scope.

% The \emph{mission statement} needs to be specific, yet flexible. Attempting to excel in multiple dimensions usually results in a product being mediocre in all of them and excellent in none, or delayed releases.

\paragraph{Risk-Driven Cycle Planning} The mission statement's aim is to expose the customer's desired result for the project.
It's likely that the customer has no knowledge of the field, so this can be done adequately in layman's terms. Using the mission statement, the team forces it's focus on a single area --- such as features, schedule, defects, or resources --- in which it must excel in the development phase.

Finally, the team fixes a set of features to implement in the next iteration and also \emph{fixes} a certain amount of time for it.

\subsubsection[Collaboration]{2. Collaboration}

This is the development phase. Here, effective collaboration between teams is key. Complex applications are not built --- they evolve. In this high information flow environment, one person or small group can't possibly \textsl{know it all}, so collaboration skills are paramount.

Collaboration with the client remains to handle minor issues or requirements adjustments, and preference resolutions.

\subsubsection[Learning]{3. Learning}

This is the quality control phase. Here fit the 

\end{document}